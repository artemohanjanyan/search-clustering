\documentclass{article}

\usepackage{amsmath}
\usepackage{amssymb}
\usepackage{amsthm}
\usepackage{mathtext}
\usepackage{mathtools}
\usepackage[T1,T2A]{fontenc}
\usepackage[utf8]{inputenc}
%\usepackage{geometry}
\usepackage[left=2cm,right=2cm,top=2cm,bottom=2cm]{geometry}
\usepackage{microtype}
\usepackage{enumitem}
\usepackage{bm}
%\usepackage{listings}
\usepackage{cancel}
\usepackage{proof}
\usepackage{epigraph}
\usepackage{titlesec}
\usepackage[dvipsnames]{xcolor}
\usepackage{stmaryrd}
\usepackage{cellspace}
\usepackage{cmll}
\usepackage{multirow}
\usepackage{booktabs}
\usepackage{tikz}
\usepackage{caption}
\usepackage{wrapfig}
\usepackage{minted}
%\usepackage{eucal}

\usepackage[hidelinks]{hyperref}

%\setmainfont[Ligatures=TeX,SmallCapsFont={Times New Roman}]{Palatino Linotype}

\usepackage[russian]{babel}
\selectlanguage{russian}

\hypersetup{%
    colorlinks=true,
    linkcolor=blue
}

\pagenumbering{gobble}

\title{Отчёт}
\author{Артем Оганджанян}
\date{}

\begin{document}

\usetikzlibrary{arrows.meta,positioning,matrix}

\maketitle

\section{\texorpdfstring{Детали реализации}{Implementation details}}
Одно из узких мест алгоритма "--- пересчёт функции сходства после слияния вершин.
Оптимально пересчитывать функцию сходства только между теми парами вершин, между которыми она могла измениться.
Мы пользуемся следующей функцией сходства:
\[
    \mathrm{Sym}(x, y) = \begin{cases}
        \frac{\displaystyle \left|L(x, y)\right|}{\displaystyle \left|L(x) \cup L(y)\right|}\text{,} &
                \text{если }\left|L(x) \cup L(y)\right| > 0\text{;} \\
        0 & \text{иначе.}
    \end{cases}
\]

Пусть мы объединяем документы D1 и D2.
Понятно, что в доле с документами изменятся только сходства с новой вершиной.
Чтобы определить пары вершин из доли с запросами, между которыми может поменяться сходство,
рассмотрим три множества вершин:
\begin{enumerate}
    \item зелёное множество вершин, которые связаны и с D1, и с D2;
    \item жёлтое множество вершин, которые связаны только с D1;
    \item розовое множество вершин, которые связаны только с D2.
\end{enumerate}

\begin{figure}[h]
    \centering
    \begin{tikzpicture}[-,every edge/.style={draw=black,thick}]
    \begin{scope}[every node/.style={circle,thick,draw},align=center]
        \node[fill=yellow] (q1) at  (0,0) {Q1};
        \node[fill=yellow] (q2) at  (2,0) {Q2};
        \node[fill=lime]   (q3) at  (4,0) {Q3};
        \node[fill=lime]   (q4) at  (6,0) {Q4};
        \node[fill=pink]   (q5) at  (8,0) {Q5};
        \node[fill=pink]   (q6) at (10,0) {Q6};

        \node              (d1) at (3,-2) {D1};
        \node              (d2) at (7,-2) {D2};
    \end{scope}

    \path [-]  (q1) edge node[anchor=north east] {1} (d1)
               (q2) edge node[anchor=east]       {2} (d1)
               (q3) edge node[anchor=south east] {4} (d1)
                    edge node[anchor=north]      {3} (d2)
               (q4) edge node[anchor=north]      {3} (d1)
                    edge node[anchor=south west] {4} (d2)
               (q5) edge node[anchor=west]       {2} (d2)
               (q6) edge node[anchor=north west] {1} (d2);
    \end{tikzpicture}
    \caption{Граф до слияния.}
    \label{diagram}
\end{figure}

Во-первых, поменяются все сходства с зелёными вершинами, так как после слияния изменятся веса инцидентных им рёбер.
Веса остальных рёбер не изменятся, однако появятся новые пары вершин из доли запросов, которые связаны с одинаковыми документами.
Множество таких пар "--- декартово произведение жёлтого и розового множеств.
Для таких пар изменится числитель из функции сходства.

\begin{figure}[h]
    \centering
    \begin{tikzpicture}[-,every edge/.style={draw=black,thick}]
    \begin{scope}[every node/.style={circle,thick,draw},align=center]
        \node[fill=yellow] (q1) at  (0,0) {Q1};
        \node[fill=yellow] (q2) at  (2,0) {Q2};
        \node[fill=lime]   (q3) at  (4,0) {Q3};
        \node[fill=lime]   (q4) at  (6,0) {Q4};
        \node[fill=pink]   (q5) at  (8,0) {Q5};
        \node[fill=pink]   (q6) at (10,0) {Q6};

        \node              (d1) at (5,-2) {D1,D2};
    \end{scope}

    \path [-] (q1) edge node[anchor=north] {1} (d1)
              (q2) edge node[anchor=south] {2} (d1)
              (q3) edge node[anchor=west]  {7} (d1)
              (q4) edge node[anchor=east]  {7} (d1)
              (q5) edge node[anchor=south] {2} (d1)
              (q6) edge node[anchor=north] {1} (d1);
    \end{tikzpicture}
    \caption{Граф после слияния.}
    \label{diagram}
\end{figure}

Для быстрого нахождения этих множеств было решено хранить граф в виде списка смежности,
в котором списки смежности для каждой из вершин отсортированы согласно индексам.
Таким образом, можно за линейное от количества связанных с D1 и D2 вершин время найти описанные множества,
и построить список смежных вершин для новой вершины.
Сделать это можно с помощью аналога алгоритма merge из merge sort.

\section{\texorpdfstring{Результаты}{Results}}
Реализованный алгоритм при запуске на больших объёмах входных данных
ожидаемо упирается в ограничения оперативной памяти и начинает долго работать.
При запуске алгоритма на миллионе случайных строк из 500k User Session Collection
3 гигабайта оперативной памяти расходуется во время начального подсчёта функции сходства.

Обработка 500000 случайных строк из 500k User Session Collection при threashold равном 0,9 использовала примерно 1,5 гигабайта памяти
и завершилась примерно за 10 минут.
201837 уникальных запросов были объединены в 105713 кластеров.
Четыре кластера содержали не меньше 1000 вершин,
55 кластеров содержали не меньше 100 вершин.
При ручном рассмотрении нескольких кластеров нашлись визуально похожие запросы,
но в больших кластерах часто между запросами не было явной связи.

Для сборки проекта нужно запустить команду
\begin{minted}{text}
$ ./gradlew installDist
\end{minted}
в корневой директории.
Послее этого в директории \texttt{build/install/search-clustering/bin/} появится исполняемый файл \texttt{search-clustering}.
Через аргументы командной строки он принимает входные файлы,
опцией \texttt{-o} задаётся выходной файл,
опцией \texttt{-threshold} задаётся порог значения функции сходства.
\begin{minted}{text}
$ ./search-clustering                                                          
Option "-o" is required
 arguments    : input files
 -o FILE      : output file
 -threshold N : threshold (default: 0.1)
$ ./search-clustering input -o output -threshold 0.9
\end{minted}

\end{document}
